\documentclass{scrreprt}

\usepackage[english,czech]{babel}
\usepackage[utf8]{inputenc}
\usepackage{graphicx}

\begin{document}

\setcounter{chapter}{11}
\chapter{Tvorba webových aplikací: Architektura webové aplikace, klientská část webové aplikace, W3C doporučení, webové skriptovací jazyky.
Grafická a strukturální stránka prezentace.}

\section{Úvod}
Současná architektura webu je založena na vztahu \textbf{klient - server}. Jediný způsob komunikace - tenký klient vysílá požadavky a server odpovídá
(změna od HTML5). 

Je možné dělit web na \textbf{statický}, \textbf{dynaický} a \textbf{webové aplikace} (s ajaxem).

Mnoho standardů je specifikováno v RFC (Request for Comments); např HTTP, HTML, SMTP. 

\section{Obsah}

\subsection{protokol - HTTP (Hypertext Transfer Protocol)}
\begin{itemize}
  \item bezestavový, textový
  \item metody *\footnote{* - bezpečné metody = jen pro čtení} - GET*, POST, HEAD* (jako get, ale poskytne pouze metadata), PUT, DELETE, trace*,
  options* (dotaz na poskytované metody), connect  
  \item hlavička GETu obsahuje - UserAgent , accept-ln, accept-encoding, cookies
  \item HTTP 1.0 vs HTTP 1.1  -  1.1 obsahuje délku zprávy
  \item stavy : 2XX - Úspěch x 3XX - Přesměrování x 4XX - Chyba klienta x 5XX - Chyba serveru  (200-ok , 403-forbidden, 404-notfound)
\end{itemize}        
    
\subsection{HTML (HyperText Markup Language)}
\begin{itemize}
  \item skupina SGML (Standard General Markup Language), XHTML skupina XML skupina SGML...XML vs SGML - párovost tagů v xml
  \item zlepšování přístupnosti - labely, dlouhé popisy, alternativní texty, thead\ldots
  \item zlepšení použitelnosti - accesskey
  \item od HTML 4 užívání CSS - oddělení obsahu a formy, kvůli robotům, vyhledávačům
  \item DOM = Document Object Model - (stromová) reprezentace dokumentu
\end{itemize}

    
\subsection{CSS (Cascading Style Sheets)}
\begin{itemize}
  \item selectors (\#-id, .-class)
  \item dědičnost (dá se porušit important)
  \item @media (styli pro tisk, handheld)
  \item je vhodné zadávat v relativních jednotkách (em, ex)
\end{itemize}


\subsection{JS (JavaScript)}
\begin{itemize}
  \item skriptovací jazyk na straně klienta, má omezená práva, dá se využít k měnění DOMu, umí reagovat na události
  \item JSON (JavaScript Object Notation) - textový formát pro výměnu dat, skládá se z kolekcí páru název/hodnota a z tříděného seznamu
  \item AJAX (Asynchronous JavaScript and XML) - provádění requestů za účelem jiným než je načítání celé stránky
  \item objektovost a dědičnost - simulace,  simulace pomocí prototypů
\end{itemize}

    
\subsection{Doporučení W3C (World Wide Web Consortium)}
World Wide Web Consortium (W3C) je mezinárodní konsorcium, jehož členové společně s veřejností vyvíjejí webové standardy. Takže jestli chcete něco
dostat do standardů, tak je dobré mít \textbf{doporučení od W3C}.

Typy HTML 4.01 DTD (Document Type Definition)
\begin{itemize}
  \item Strict includes all elements and attributes that have not been deprecated or do not appear in frameset documents
  \item Transitional = strict + deprecated elements and attributes (most of which concern visual presentation)
  \item Frameset = transitional + frames
\end{itemize}    
        

\section{Zajímavosti}
Kdo by si přál konstanty v CSS?

\subsection{novinky v HTML 5}
\begin{itemize}
  \item přibilo - canvas, audio, video
  \item Web Sockets - obousměrná komunikace 
  \item Web Storage - key-value databáze, IndexedDB - robustní indexovaná databáze
  \item Web Workers - obdoba vlákna
  \item Native Drag \& Drop
\end{itemize}

\subsection{knihovna jQuery}
mimo pěkné funkce umožňuje používat selectory v JS

\section{Příloha}
\subsection{JS - prototipování + dědičnost}
\begin{verbatim}
		function vesmirny_objekt() {
			this.soustava = "slunecni";
		}
		function planeta(pocet) {
			this.pocet_mesicu = pocet
		}
		planeta.prototype = new vesmirny_objekt();
		zeme = new planeta(3);
		alert(zeme.pocet_mesicu);
		alert(zeme.soustava);
\end{verbatim}


\end{document}