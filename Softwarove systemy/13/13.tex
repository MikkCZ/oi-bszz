\documentclass{scrreprt}

\usepackage[english,czech]{babel}
\usepackage[utf8]{inputenc}
\usepackage{graphicx}

\begin{document}

\setcounter{chapter}{12}
\chapter{Jazyk PHP, server-klient interakce. Šablony, MVC, frameworky, oddělení prezentační a aplikační logiky. Webové služby, AJAX.}

\section{Úvod}
Současná architektura webu je založena na vztahu \textbf{klient - server}. Jediný způsob komunikace - tenký klient vysílá požadavky a server odpovídá
(změna od HTML5). 

Je možné dělit web na \textbf{statický}, \textbf{dynaický} a \textbf{webové aplikace} (s ajaxem).

Mnoho standardů je specifikováno v RFC (Request for Comments); např HTTP, HTML, SMTP. 

\section{Obsah}

\subsection{PHP (Personal Home Page)}
\begin{itemize}
  \item pole jsou asociativní, tedy ve skutečnosti se jedná o (hašovací) tabulky, které ukládají páry klíč -> hodnota.
  \item slabě dynamicky typovaný
  \item Imperativní (procedurální) a skriptovací jazyk, od PHP 5.0 podpora objektů
  \item C-like syntax
  \item Od PHP 5.3.0 - namespaces
  \item magické metody - \_\_ , \_\_get \_\_set \_\_isset \_\_unset \_\_clone (shallow copy)
  \item má modifikátory viditelnosti, final
  \item typi hinting u parametrů funkcí (kontrola typů), neuspokojení může způsobit fatal error
  \item Magické konstanty - \_\_LINE\_\_, \_\_FILE\_\_, \_\_DIR\_\_, \_\_FUNCTION\_\_, \_\_CLASS\_\_, \_\_METHOD\_\_, \_\_NAMESPACE\_\_
  \item velikou výhodou ve široká podpora a komunita
\end{itemize}


\subsection{bezpečnost s PHP}
\begin{description}
  \item[Cross-side scripting] narušení html stránky pomocí uživatelského vstupu (html tagy,JS). Dá se bojovat funkcí \emph{htmlspecialchars}. 
  \item[SQL injection] ovlivňování SQL databáze nezamýšlený způsobem pomocí uživatelského vstupu. Dá se bojovat pomocí \emph{prepareStatement} nebo
  \emph{escape_string}. Někdo využívá na funkci \emph{magic_quotes}, která automaticky escapuje veškeré parametry GET a POST požadavků.
\end{description}


\subsection{MVC}
\begin{itemize}
  \item výhody - dobrá dělba práce + v aplikaci se častěji mění view + lehce umožňuje předkompilovat části kódu, takže zrychlení (a odstranění
  komentářů)
  \item Page Controller(URL má vlastní controller) X Front Controller(jeden hlavní controller) X Composite View(stránka je složená z více view)
  \item super šablony Smarty
  \item celé MVC - Zend, Nette  
\end{itemize}


\subsection{frameworks}
\begin{itemize}
  \item Doctrine - ORM
  \item ZEND, NETTE  - řeší lokalizaci, MVC, strukturu projektu, formuláře, ajax, \ldots
\end{itemize}

\subsection{Webové služby}
Toto téma je již zpracováno.

\subsection{Ajax}
Asynchronous JavaScript and XML - provádění requestů za účelem jiným než je načítání celé stránky (úprava DOMu na základě nových dat ze serveru,
asynchroní odesílání dat).

\end{document}

